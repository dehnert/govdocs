% A quick summary for the chair of a meeting.
% typo fixes and formatting tweaks Apr 96.
\documentstyle{article}
\pagestyle{empty}
\oddsidemargin 0in
\textwidth 6.5in

\headsep=0pt \headheight=0pt  \topmargin=0in
\textheight=9.4in
\voffset=-0.4in
\begin{document}

\title{Robert's Rules: a quick summary for the chair}
\author{by Stephen Gildea}
\date{15 December 1987}
\maketitle
\thispagestyle{empty}

Club meetings should be as informal as possible.  One of the goals of
Robert's Rules is to slow things down enough that everyone has a
chance to participate and no one gets railroaded.  However, if the
meeting is moving along smoothly and everyone is happy, don't let the
formalities get in the way.  My rule of thumb is to be as formal as
the members insist.  That is, if anyone insists you do it by the book,
then do, otherwise keep things moving.

\section{Motions}

Steps to passing a motion:

\begin{enumerate}
\item
Someone makes a motion: ``I move \ldots''

All motions must be seconded.
A member can say ``I second the motion'' or simply ``Second.''
If there is not an immediate second,
ask for it: ``Is there a second?''  If there isn't, the motion is ignored.


\item
Discussion (debate).  Anyone wishing to speak must be recognized by the chair
first.  Members signal that they wish to be recognized by raising
their hands.  (In some places they stand up instead.)

\item
End of debate.  If debate is winding down, the chair may say ``Are you
ready to vote?''  (Or, more formally, ``Are you ready for the
question?'')  If there are no objections, or no one wants to speak
further, proceed to voting.

If some members want to continue discussion, the group must vote by a
2/3 majority to end debate.  A member may force this vote with a
motion called {\em the previous question:\/}  ``I move the previous question.''
The chair must then immediately call for the vote.  Normally we don't
have to get this formal.

\item
Voting.  In order of increasing formality: a voice vote (ayes and
nos), hand count, secret (paper) ballot.

The chair says ``We are voting on\ldots.  All in favor raise their
right hand (say `aye').''  Note the count.  ``All opposed raise their
right hand (say `no').''  Note the count.  If taking a hand count,
you must ask for abstentions ``Abstentions.''  The sum of the three
numbers should be the number of people in the room.  When calculating
majorities, don't count the abstentions.

Always announce the result of the vote: ``The motion passes, 17 to 4.''

\end{enumerate}

\section{White Ballot}

The white ballot is also sometimes called {\em vote by acclamation\/}
and is a time-saving device.  The idea is that if no one objects to a
motion, it passes without a formal vote.
In my experience it takes longer to explain the process to the meeting
then just to hold a quick vote.  However, someone may bring it up, so
you have to know what to do. 

Member: ``I move a white ballot.''  Or: ``I move we adopt the motion
by acclamation.''

The chair: ``A white ballot has been moved.''

The first time this happens you have to explain what's going on:
``If no one objects, this motion will
pass without a formal vote.  If any single person objects, we will
vote normally.''

Then: ``Are there any objections to passing (state the motion)?''
Pause to give members a chance to object.  Then: ``Hearing no
objections\ldots'' pause again briefly ``\ldots the motion passes.''

\end{document}
