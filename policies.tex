\documentclass{bylaws}
\title{Tech Squares Standing Policies}
\date{2015}

\newcommand{\sptimes}[2]{\textit{Adopted #1. Expires #2.}}
\renewcommand{\thesection}{Standing Policy \arabic{section}.}

\begin{document}

\maketitle


\policy{Membership}
\sptimes{Spring 2015}{Spring 2018}

\section At their discretion, the Executive Committee may grant membership to prospective members who have not gone through the class. Such prospective members must dance club level and have attended at least four of fifteen consecutive club dances.
\section Club membership does not expire, though a club member may relinquish membership.
\section In general, voting members shall be a member who has attended at least four recent club dances. For this purpose, a ``recent club dance'' shall be any since the fifteenth most recent non-summer (June, July, and August) club dance (inclusive).


\policy{Elections and Appointments}
\sptimes{Spring 2015}{Spring 2018}

\subarticle{Elections}

\subsection The elected positions shall be elected in the order specified in the constitution.
\subsection For each position,
\duty Each candidate shall have a chance to state a platform and answer questions.
\duty The club shall have an opportunity to discuss in private, including for uncontested positions.
\duty The club may call individual candidates back in for additional questions.
\duty A majority vote is required to elect a candidate.
\subsection After electing all other candidates, Members-at-Large will be elected to fill out the EC. Some Members-at-Large may be need to be students. All student-only Member-at-Large positions will be elected simultaneously first, followed by all remaining Members-at-Large. During Member-at-Large elections, each club member votes for a number of candidates up to the number of positions being filled.

\subarticle{Appointments}

\subsection The Executive Committee may appoint additional officers, including but not limited to those required by the constitution or standing policy.
\subsection The term of appointed officers shall be no longer than the term of the appointing EC.
\subsection The EC shall solicit suggestions from the club for appointed positions.
\subsection The incoming EC is encouraged to select new appointed officers prior to taking office, so that the EC and appointed officers take office simultaneously.


\policy{Meetings}
\sptimes{Spring 2015}{Spring 2018}

\subarticle{Executive Committee}

\subsection Executive Committee members shall attend Executive Committee meetings.
\subsection Executive Committee meetings shall be announced to the Executive Committee members at least 24 hours in advance.
\subsection Meetings shall be open to the club except when sensitive or confidential information will be discussed. Open meetings shall be announced to the club in advance.
\subsection Quorum shall be a majority of the EC.

\subarticle{Club Meetings}
\subsection Club meeting shall be held during or immediately before or after regular weekly dances.
\subsection Proxies are not allowed at club meetings.
\subsection Club meetings shall be announced to the club two weeks in advance. This does not apply to meetings rescheduled after postponement due to cancelled dances or lack of quorum.
\subsection Club meetings may be called by the EC, or by petition of five voting members.


\policy{Class}
\sptimes{Spring 2015}{Spring 2018}
\section The Class shall generally be open to everyone.
\section The Class Coordinator, after notifying the EC, may remove people from the class if they are interfering with the class or preventing other class members from learning.
\section At the discretion of the EC, up to one class per year may be open to MIT students only. % TODO: check language
\section The EC must approve the list of class graduates. The Class Coordinator shall provide the EC with a list of class members, and provide a recommendation of which should graduate, which should not, and which deserve discussion.
\section A class member must dance club level or be expected to dance club level soon to graduate. Attending 80\% of the class is often a sign that a class member will dance club level.


\policy{Duties}
\sptimes{Spring 2015}{Spring 2017}

\subarticle{Elected officers}

\subsection The President and Vice President shall
\duty be responsible for all club operations
\duty oversee all other officers
\duty call and chair meetings of the club and the Executive Committee
\duty be a member of all committees
\duty appoint temporary officers as needed
\duty liaise with the ASA and other outside bodies as needed

\subsection The Treasurer and Vice Treasurer shall
\duty be responsible for the club's financial transactions
\duty ensure that gate fees and other accounts receivable are collected and deposited
\duty ensure that caller fees and other accounts payable are disbursed
\duty keep financial records of the club
\duty report annually to the club on the financial status of the club

\subsection The Class Coordinator shall
\duty organize the class
\duty take admission from class members
\duty take attendance of class members
\duty maintain lists of calls and definitions taught in the class
\duty run class meetings
\duty order club badges for new graduates and other club members
\duty add new graduates to the club roster and mailing lists
\duty organize graduation
\duty act as a liaison between the class and club

\subsection The Publicity Coordinator shall
\duty publicize the start of Tech Squares classes and other events, as appropriate
\duty manage the club's social media presence
\duty design and distribute posters
\duty prepare succinct announcements to be read at club dances as needed
\duty send weekly emails to the club informing them of the location and schedule of dances

\subsection The Members-at-Large have no pre-defined tasks. They are encourage to choose specific ways to help the club during their term. Preferably, they will consider this prior to election and include their preferences in their platform.

\subarticle{Appointed officers}
\subsection The Booking Director shall
\duty book contracts with callers and cuers for Tech Squares events
\duty make arrangements for temporary callers or cuers when the club caller or cuer is unavailable
\duty confirm arrangements with the caller and cuer shortly before each event
\subsection The Rooming Director shall reserve rooms for club dances and other club functions.
\subsection The Rounds Coordinators shall manage the club's round dancing program.

\subarticle{Club Jobs}
\subsection Club dances shall be run by the officers and other club members.
\subsection Each dance shall have a Dance Supervisor (possibly the President) with overall responsibility for ensuring the smooth functioning of the dance.
\subsection A schedule assigning duties to these people will be published in advance of each dance. Such duties may include
\duty opening and setting up the hall
\duty assisting the caller and cuer, including helping them find the hall
\duty coordinating refreshments
\duty taking admission and attendance of club members and guests
\duty making announcements at dances
\duty tallying money at the end of the dance, and arranging a deposit if needed
\duty ensuring the hall is closed properly after dances

\subarticle{Additional duties}
\subsection The EC has several additional collective responsibilities. EC members, especially Members-at-Large, should assume responsibility for some of these jobs, and the EC should appoint officers to do the other jobs.
\subsection These duties include
\duty \textbf{Recording secretary}: taking minutes at Club and Executive Committee meetings
\duty \textbf{Corresponding secretary}: handling club correspondence
\duty \textbf{Roster manager}: maintaining a list of members and preparing lists of voting members as needed
\duty \textbf{Club photographer}: taking photos and videos of club events and adding them to a club photo gallery
\duty \textbf{Webmaster}: updating the website when other EC members are unable or do not wish to do so themselves
\duty \textbf{Dance supervisor} for events (such as Saturday dances) for which somebody more specific than the President is desired


\policy{Adopting and Amending Standing Policies}
\sptimes{Spring 2015}{Spring 2016}
\section Standing policies may be adopted or amended by majority vote of the club.
\section Standing policies may be adopted or amended by majority vote of the EC. The EC shall inform the club of any such changes it makes to the standing policies, and no changes shall take effect until two weeks after said notification. Upon petition by 20\% of voting members of the club, or majority vote of the EC, any EC change shall be delayed until addressed at a club meeting. Alternatively, the EC may choose to discard any change rather than bringing it to club vote.
\section Each standing policy shall have an expiration date. Unless otherwise specified, this shall be the end of the term three years after adoption or re-approval.
\section For each standing policy, early in the term that it will expire, the EC shall publicize and hold an open EC meeting to discuss the policy and any needed amendments. The EC shall decide whether to re-approve it (possibly with amendments), remove it, or refer it to the club for further discussion.

\end{document}
