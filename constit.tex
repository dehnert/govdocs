% TECH SQUARES CONSTITUTION
% Started by Stephen Gildea 24 July 82, most of the work done
% Spring 1984.
% Many people helped me with this.  For a list of acknowledgements,
% see the file "notes".
%
% Constitution ratified by a vote of 74 to 7 with 1 abstention
%
% 13 August 2002
%    Re-inserted Art II Sec 4, "At least one-half of the voting
%    membership shall be MIT students."  Had been removed Jan 1988
%    when the ASA's new constitution stopped requiring it, but in 
%    May 2002 they gave us grief for not having this.
%    Amendment passed 19-0-1 (that is, 1 abstention).
%
% 16 August 1994
%    1. Elections: add optional fall elections, so officers have an
%       out after 1 term.  Spring officers take office in June, since
%       class now often not finished by May.
%    2. Executive Committee is elected (not appointed) officers only
%    3. club employees may be appointed offiers
%    4. reduced attendance requirement for becoming a member to similar
%	to keeping voting membership
%    5. class is 14 weeks, not 12, to match current practice
%    6. grammar fixes
%
% 1 March 1988
%    the entire officer structure changed
%
% 19 January 1988
%    some minor changes that I wanted to see made were ratified.
%
% gildea December 1987:
%    Formatting converted from troff to LaTeX.
%    Several amendments proposed
%
\documentclass{bylaws}[2002/09/05]
\title{Tech Squares Constitution}
\date{13 Aug 2002}
\begin{document}

\maketitle

\begin{history}
Adopted 22 May 1984
Amended 19 January 1988
Amended 1 March 1988
Amended 16 August 1994
Amended 13 August 2002
\end{history}


\article{Name and Purpose}
\section
This organization shall be called
Tech Squares,
the Square and Round Dance Club of the
Massachusetts Institute of Technology,
hereinafter referred to as
{\it the club\/} and {\it MIT,\/}
respectively.
\section
The club shall regularly hold modern Western-style square dances.
\section
The club shall hold classes to
introduce beginners to square dancing.

\article{Membership}
\section
Upon graduation from the class a person becomes a
{\it club member.\/}
\section
A person who has not graduated from the class must dance club
level to become a club member.
There may be other requirements as stated in the Bylaws.
\section
Any club member who
meets the attendance requirements specified in the Bylaws
shall be a {\it voting member.\/}
% \begin{comment}section removed January 1988, re-inserted August 2002
% \end{comment}
\section
At least one-half of the voting membership shall be MIT students.
\section
Membership shall include the right to wear the club badge.
No one but club members may wear the club badge.
\section
Only voting members may vote at club meetings or hold office.

\article{The Class}
\section
The club shall hold a class during each Fall and Spring academic
term at MIT to introduce beginners to square dancing.
\section
The class shall be open to anyone.
\section
The Bylaws shall state criteria for graduation from the class.
\section
The class shall be coordinated by the Class Coordinator.

\article{Officers}
\section
The elected officers of Tech Squares shall be the
President, Treasurer, Secretary, Publicity Coordinator, and
Class Coordinator,
ranked in that order.
They shall be elected as specified in the Bylaws.
\section
The term for officers elected in the spring shall be seven
months, with an option to continue in office for an additional five
months without re-election.  Officers elected in the fall, if any,
shall hold office for five months.
\section
The President shall be responsible for
the overall operation of the club and be
its official representative to the square dance
community and to all callers and cuers hired by the club.
\section
The Treasurer shall be directly accountable
for the finances of the club.
\section
The Secretary shall handle all club correspondence and be responsible
for minutes at all club and Executive Committee meetings.
\section
The Class Coordinator shall
chair the Class Committee, which shall
be responsible for the organization of the
class, and shall appoint Class Assistants to assist in
administering the class.
\section
The Publicity Coordinator shall
chair the Publicity Committee, which shall
be responsible for all club and class
publicity, and shall appoint Publicity Assistants to help.
\section
The Executive Committee shall appoint a Booking Director, Rooming
Director, Banner Raid Director, and Archivist/Historian.  Their term of
office shall each be one year.
\section
The Executive Committee shall appoint two Members-at-Large and
other officers as specified in the Bylaws or as deemed appropriate.
Their term of office shall not exceed one year.
The Executive Committee shall solicit suggestions from the
club for appointed positions.
\section
The Members-at-Large shall serve as
liaisons between the officers and other club members.
The Members-at-Large shall be appointed by the Executive Committee
three times a year: at the beginning of the Fall, Spring, and
Summer terms at MIT.
Members-at-Large may not serve more than two consecutive terms.
\section
No person may hold two or more elected offices concurrently.
A person may hold an appointed position concurrently with any
other position, including another appointed position.
\section
Upon petition by at least 20\%
of the voting members of the club or one-half the members of the
Executive Committee, the Executive Committee shall call a club meeting
to discuss the removal of an elected officer.  The officer may be
removed by a vote of three-fourths of those present and voting.

\article{The Executive Committee}
\section
The club
shall have an Executive Committee which shall be
responsible for the affairs of the club between
club meetings and is empowered to make such decisions as it feels
are necessary to efficiently carry on the business of the club,
except as otherwise governed by the Constitution or
the Bylaws.
\section
The Executive Committee shall be comprised of the five
elected officers.
Executive Committee members are encouraged to attend Executive
Committee meetings.
\section
Officers shall have additional and more specific
duties as defined in the Bylaws
and as the Executive Committee finds necessary for the functioning of
the club.
\section
The Executive Committee shall regularly hold meetings with
at least one meeting between each club meeting.
Additional meetings shall be held
within one week of a request by either
the President or by any two members of the Executive Committee.
\section
Each elected officer shall have one vote on the Executive Committee.
\section
A majority vote of the voting members of the Executive Committee shall
be necessary for any decisions to be made by it.

\article{Meetings}
\section
The club shall regularly hold meetings, called
{\it club meetings,\/}
to review the club status and conduct club business.
\section
There shall be at least one club meeting
during each of the Spring and Fall terms at MIT\null.
Additional meetings may be called by the Executive Committee
or by petition of any five voting members.
\section
Club meetings shall be open to all club members except as otherwise
noted in the Constitution or Bylaws.
\section
A quorum of the club shall consist of one-third of the
voting membership.
At club meetings, no business other than reports and discussions shall
be carried on unless a quorum is present.
\section
The most recent edition of
{\it Robert's Rules of Order, Newly Revised\/}
shall decide all questions of procedure not specifically covered in the
Constitution or in the Bylaws.
\section
All club meetings shall be chaired by the highest ranking elected officer present.
\section
Decisions of any club meeting shall be binding upon the Executive Committee
and the club officers.

\article{Dances}
\section
The club shall hold weekly dances, called
{\it club dances,\/}
year-round allowing exceptions for holidays and similar occasions.
\section
In addition to club dances, the club shall hold other dances, called
{\it monthly dances,\/}
at the discretion of the officers.
\section
The club shall choose a
{\it club level\/}
for its square dances.  Club dances
and monthly dances shall be at club level, except as required for the class.
\section
The club may hold dances and workshops at other levels.

\article{Club Employees}
\section
The club shall hire two permanent employees to provide calling and cueing
at club dances: a square dance caller, called the
{\it club caller,\/}
and a round dance cuer, called the
{\it club cuer.\/}
\section
The club caller and the club cuer shall be members, but
not voting members, of the club for the term of their employment.  They may
wear the club badge.  They may attend club meetings only at the invitation of the Executive Committee.
\section
The selection of a club caller or club cuer must be approved by the club at
a club meeting.
\section
No person may be an elected club officer and a club employee concurrently.
\section
The Executive Committee may hire callers and cuers for monthly dances.
Such employees do not become club members by fact of their employment
by the club.
\section
Except as mentioned in this article, the Executive Committee may not hire
employees without the prior consent of the club at a club meeting.

\article{Amendment}
\section
This Constitution may be amended at a club meeting by a two-thirds vote of
those present and voting.
\section
A proposed amendment to the Constitution must be submitted in writing to
the Secretary.  The Secretary shall distribute copies of the proposed
amendment to the club at least four weeks before the club meeting at which
the amendment will be voted on.

\article{Law}
\section
The club shall be a recognized MIT student organization.
\section
This Constitution is the governing law of the club, except where it
conflicts with the rules of MIT.
\section
The club shall abide by the rules and regulations of the MIT Association of
Student Activities, hereinafter referred to as the
{\it ASA.\/}
This Constitution shall be subject to the approval of the ASA.
\section
The obligations of the club shall
remain in effect unless specifically negated.

\article{Ratification}
\section
This Constitution shall be ratified when approved by a two-thirds vote of
those club members voting.  A provision for absentee ballots shall be made.
At least four weeks' notice shall be given of the club meeting at which
the ratification shall be voted on.

\end{document}
