% roberts.tex -- A short summary of Roberts Rules for meeting attendees.
% Time-stamp: <96/04/30 17:06:26 gildea>
%
% It was 1 page when Larry gave it to me, but then I expanded it to
% two pages.
% Don't remember now (1996) which % Larry the previous sentence refers
% to.  Larry Weinstein?
% Appears to have been written 18 Dec 1987.  Minor typos fixed April 1996.
\documentstyle[erl]{article}
\oddsidemargin 0pt
\textwidth 6.5in
\pagestyle{empty}
\begin{document}
\begin{centering}
{\Large
Parliamentary Procedure \\
{\normalsize as defined by}\\
Robert's Rules of Order \\
{\normalsize a summary for meeting attendees}\\[3ex]
}
\end{centering}

The Chairman runs the meeting.  No one may speak unless recognized by
the Chairman.  (There is one exception to this rule; see Point of
Order below.)  If you wish to speak, raise your hand and wait quietly
until recognized by the Chairman.

\bigskip
Here are the steps of passing a motion:

\begin{itemize}
\item
Making the motion.  Making a motion requires two people.  The first
person states the motion (by saying ``I move that $\ldots$''), and the
other person seconds it (by saying ``Second'').
A motion dies immediately if there is no second.
Motions submitted by committees do not require a second.
\item
Discussion.  If the motion is debatable, the members have a chance to
discuss the motion before voting on it.  All speakers must be first
recognized by the Chairman.  If the speaker is willing to answer
questions, the Chairman, not the speaker, calls on people who have
questions.  The speech plus the question period must not exceed the
time limit set for speeches.
Unless otherwise specified, all speakers may speak twice on a given
question for ten minutes each time.
Questions must be true questions
pertaining to the speaker's speech.  If you wish to rebut the
speaker's points, you should make a separate speech on your own time.
\item
The end of discussion.  If the Chairman notes that no one wants to
speak any more, the Chair may ask, ``Are you ready to vote?''  The Chair
must then pause for a moment to see if anyone still wants to speak.
If anyone does, the Chairman must let them speak.  If no one wants to
speak, the Chairman moves on to the voting.  If there are
people who want to speak, a vote can still be forced by using the
motion Previous Question (see below).
\item
The vote.  There are several ways of voting: white ballot, voice vote,
hand count, and paper ballot.  Probably the only one that needs
explaining is the white ballot.  The white ballot is a time-saving
method of passing simple motions that everyone agrees on without
taking a formal vote.  The procedure is this:  when the group is ready
to vote, a member may say, ``I move white ballot.''  The motion must
be seconded.  Then the Chairman says, ``There has been a call for a
white ballot; hearing no objections $\ldots$ (the Chairman pauses here
to listen for objections) $\ldots$ the motion passes.''  If any one
person objects (by saying ``I object'') then a true vote must be
taken.
\end{itemize}

There are different types of motions of different priorities.  Any
motion that actually does something is called a Main Motion.  The
other types of motions affect how the meeting is run.

The highest priority pending motion is the ``motion on the floor.''
All discussion must be related to it.  The consideration of a motion
can be interrupted if a higher-priority motion is made.  After the
higher-priority motion is voted on or otherwise disposed of, the
discussion returns to the motion that was interrupted.

\bigskip
\begin{tabular}{c|l|c|c|c|c}
Precedence & Motion & Second? & Amend? & Debate? & Needs \\ \hline
highest & Adjourn & YES & no & no & majority \\
$\uparrow$ & Point of Information or Order & no & no & no & --- \\
$\cdot$ & Previous Question (end debate) & YES & no & no & 2/3 \\
$\cdot$ & Limit Debate & YES & YES & no & 2/3 \\
$\cdot$ & Postpone to a Specific Time & YES & YES & YES & majority \\
$\downarrow$ & Amend & YES & YES & YES & majority \\
lowest & Main Motion & YES & YES & YES & majority
\end{tabular}

\begin{list}{}{}
\item[{\bf Main Motion}]
This is the main item of discussion.  Most motions are Main Motions.
\item[{\bf Amend}]
This can be used to change any amendable motion.  The amendment is acted on
before the motion itself.  If the amendment passes, then when
discussion returns to the amended motion, it is the amended version
that is considered.
\item[{\bf Postpone}]
Delays the discussion and voting of a motion to a later time.
\item[{\bf Limit Debate}]
Initially all speeches are limited to being 10 minutes long.  This
motion can be used to change that.  Since it affects the ability of
members to speak, it requires a 2/3 vote.
\item[{\bf Previous Question}]
This motion ends debate (on whatever was being debated) and forces an
immediate vote.  Since it affects the ability of members to speak, it
requires a 2/3 vote.  It is not debatable itself.
\item[{\bf Point of Order}]
A Point of Order or Point of Information can be raised at any time.
It is not really a motion (so it doesn't require a second), but just a way of
correcting an incorrect statement or procedure.  This is an exception
to the rule that members must be silent unless recognized by the
Chairman; a member wishing to raise a Point of Order may say, ``Point
of Order'' before being recognized by the Chairman.  However, the point
itself cannot be said until the Chairman recognizes the member.
\item[{\bf Adjourn}]
A motion to adjourn is always allowable.  If it passes, it ends the
meeting immediately.
\end{list}
\end{document}
