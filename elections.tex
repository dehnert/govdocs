% elections.tex
% How to run an election at Tech Squares -- notes for the President
% (A typeset version of elections.howtorun, which is now obsolete.)
% Time-stamp: <2003-08-14 23:47:20 gildea>
% $Id$
% Written April 1996 for Linda Resnick, who was President that term.
\documentclass{article}
% need page numbers now that we've gone to three pages
%\pagestyle{empty}

\usepackage{times}
\usepackage[ps2pdf,pdfusetitle]{hyperref}
\hypersetup{pdftitle=Tech Squares Elections}
%\hypersetup{pdfpagemode=UseNone} %no outline

\special{header=hp-printer-duplex.ps}

% Don't print the section number, but do have hyperref include it in
% the outline.
%\setcounter{secnumdepth}{0}
\makeatletter
\renewcommand{\@seccntformat}[1]{} %clobber the section number
\makeatother

\headheight 0pt
\headsep 0pt
\topmargin 0pt
\textheight 9in

\parskip .15\baselineskip plus 1pt

\oddsidemargin .25in
\textwidth 6in

% like quote env, but tighter
\newenvironment{chairsays}
               {\parskip 0pt \par %force linebreak in latex2html
                \begin{quote}
                 \parskip 0pt \bf}
               {\end{quote}}


\title{How to run an election at Tech Squares \\
        \Large Notes for the President}         %second line smaller
\author{Stephen Gildea}
\date{April 1997}

\begin{document} %--------------------------------------------------

\maketitle

Before beginning, ensure the Secretary or a designee is ready to record
the outcome of all votes and otherwise record any business of the meeting.

The meeting has these parts:
\begin{enumerate}
\setlength{\parskip}{0pt}\setlength{\itemsep}{0pt}
\item
explain the meeting and voting requirements
\item
elect each officer
\item
adjourn meeting
\end{enumerate}

The election for each office has these parts:
\begin{enumerate}
\setlength{\parskip}{0pt}\setlength{\itemsep}{0pt}
\item
explanation of the office to be elected
\item
nominations
\item
discussion of candidates
\item
election
\end{enumerate}

The officers are elected in this order: President, Treasurer,
Secretary, Publicity Coordinator, and Class Coordinator.

\section{Explanation of Procedure}

\begin{chairsays}
For each office, the office will be described, there will be
nominations, and we will take a hand vote.

The offices are
President, Treasurer,
Secretary, Publicity Coordinator, and Class Coordinator.
\end{chairsays}

\section{Voting Requirements}

All voting members of Tech Squares may vote.  From the Bylaws:

\begin{chairsays}
Voting members shall have attended at least five club or monthly dances in
the last four months \ldots .
\end{chairsays}

\section{Explanation of the office}

Several ways to explain the office.  Reading the formal description
from the club Bylaws might be useful.  Or the current holder of the
office or the meeting chair can just explain in their own words.
Ask for questions.

\section{Nominations}

Prior to the meeting there should have been a Nominating Committee
ensuring that there is at least one candidate for each office.
Announce these candidates.
(If you have only a reluctant backup candidate, you might want to hold
off announcing their name to see if anyone else volunteers.
The meeting Chair may nominate anyone at any time.)
Invite additional nominations from the floor:
\begin{chairsays}
Any voting member may nominate anyone, including themselves.

Are there any (other) nominations or volunteers?
\end{chairsays}

Nominations do not need to be seconded.
The nominee should be asked if they are willing to serve:
\begin{chairsays}
{\em Candidate\/}, do you accept the nomination?
\end{chairsays}

Repeat all nominations:
\begin{chairsays}
{\em Candidate\/} is nominated.

Are there any further nominations?
\end{chairsays}

When no one wants to make any additional nominations, close
nominations:

\begin{chairsays}
Are there any objections to closing nominations for {\em (this office)\/}?
{\em (Pause briefly.)}

Hearing no objections, \ldots {\em (pause briefly again)\/} \ldots
nominations are closed.
\end{chairsays}

Some member may move to close nominations before you suggest it.  If
there is likely to be general agreement to close nominations, do not
let this motion force you into taking the time for a vote.  Just
proceed into the actions of the previous paragraph.  If someone
objects, you need a 2/3 vote.

\section{Discussion of Candidates}

If there is only one candidate, this part of the election can be very
brief.  Ask,

\begin{chairsays}
Are there any questions for the candidate(s)?
\end{chairsays}

The Chair (not any of the candidates) recognizes any member who stands
up or raises their hand.  The Chair then gives the floor to each of
the candidates to respond.  If the question was directed to a specific
candidate, they may answer first, but all candidates get a chance to
respond to all questions.

The Chair may ask the candidates to leave the room for further
discussion and voting.  We normally don't do this unless there is more
than one candidate.  Always be sensitive to any members trying to
signal that they would like additional discussion without the
candidate(s) being present.

When discussion has died down, call the vote:
\begin{chairsays}
Are you ready to vote?
{\em (Pause briefly.)}

Hearing no objections, \ldots {\em (pause briefly again)} \ldots
discussion is closed.
\end{chairsays}

Again, some member may ``move to close discussion'' or ``move the
previous question.''  Unless some members wish to continue discussion,
don't take the time to recognize the motion and call a vote on it.
Just proceed as above.  If a vote on ending discussion is
required, it must pass by 2/3, not simple majority.  Therefore a 
voice vote (with ayes and nos) cannot be used; use a hand count.

\section{Election (one candidate)}

Frequently, in the case of an uncontested election, some helpful
person will call for a white ballot, election by acclamation, or
similar procedure.
I am
opposed to such supposed time-saving procedures for elections.  First,
they don't really save time; it takes as long to explain what is going
on to everyone else as it does to just hold a simple hand vote.
Second, they add unnecessary parliamentary procedure.  And third, I
like to see people actually vote; it's the American way, and it gives
a more visible mandate to the new officers.
(But note that this memo shows doing all other business without votes.)
So if an election by acclamation is moved, say firmly:

\begin{chairsays}
We will use a hand vote.
\end{chairsays}
and proceed to do so.

State what the group is voting on and call for votes.  It will usually
not be necessary to actually count hands to determine the outcome of
the vote.

\begin{chairsays}
We are voting to elect a {\em (name of office)}.

All in favor of {\em (candidate)\/} raise their right hand.
{\em \ldots}
Thank you.

All those opposed.
{\em \ldots}
Thank you.
\end{chairsays}

Announce the results:

\begin{chairsays}
{\em Candidate\/} is elected.
\end{chairsays}

\pagebreak[3]

\section{Election (multiple candidates)}

State what the group is voting on and call for votes:

\begin{chairsays}
We are voting to elect a {\em (name of office)}.

All in favor of {\em (first candidate)\/} raise their right hand.
{\em (Count hands if necessary.)}
Thank you.

All in favor of {\em (second candidate)\/} raise their right hand.
{\em (Count hands if necessary.)}
Thank you.

{\em Etc.}
\end{chairsays}

Announce the results.  State who is elected:

\begin{chairsays}
{\em Candidate X\/} received 23 votes.

{\em Candidate Y\/} received 9 votes.

{\em Candidate X\/} is elected.
\end{chairsays}

If someone asks for it or you feel it advisable, do a secret vote with
paper ballots.

You are now finished with one office.  Proceed to the next office.
Fortunately, this often takes less time to do than to read about.

\section{Adjourning}

After disposing of all business on the agenda (probably just
elections, but possibly also including a Treasurer's report),
adjourn the meeting.  If no one moves to adjourn, prompt them:

\begin{chairsays}
I \emph{(or \emph{``The Chair''})} will entertain a motion to adjourn.

{\em Member:\/} So moved.
\end{chairsays}

It is usually not necessary to prompt for a second, but if so, ask:

\begin{chairsays}
Is there a second?

{\em Member:\/} Seconded.
\end{chairsays}

Since a motion to adjourn cannot be debated, proceed immedately to act
on it efficiently:

\begin{chairsays}
If there are no objections, this meeting will be adjourned.
{\em (Pause briefly.)}

Hearing no objections, \ldots {\em (pause again)} \ldots this meeting
is adjourned.
\end{chairsays}

For more general info on meeting procedures, I have written two largely
overlapping memos called ``Parliamentary Procedure'' (for members) and
``Robert's Rules'' (for the chair).

\newpage

\section{Summary of What the Chair Says}

\begin{chairsays}
For each office, the office will be described, there will be
nominations, and we will take a hand vote.

The offices are
President, Treasurer,
Secretary, Publicity Coordinator, and Class Coordinator.
\end{chairsays}

\begin{chairsays}
Voting members shall have attended at least five club or monthly dances in
the last four months.
\end{chairsays}

\begin{chairsays}
Any voting member may nominate anyone, including themselves.

Are there any (other) nominations or volunteers?
\end{chairsays}

\begin{chairsays}
{\em Candidate\/}, do you accept the nomination?

{\em Candidate\/} is nominated.

Are there any further nominations?
\end{chairsays}


\begin{chairsays}
Are there any objections to closing nominations for {\em (this office)\/}?
{\em (Pause briefly.)}

Hearing no objections, \ldots {\em (pause briefly again)\/} \ldots
nominations are closed.
\end{chairsays}

\begin{chairsays}
Are there any questions for the candidate(s)?
\end{chairsays}

\begin{chairsays}
Are you ready to vote?
{\em (Pause briefly.)}

Hearing no objections, \ldots {\em (pause briefly again)} \ldots
discussion is closed.
\end{chairsays}

\begin{chairsays}
We are voting to elect a {\em (name of office)}.

All in favor of {\em (candidate)\/} raise their right hand.
{\em \ldots}
Thank you.

All those opposed.
{\em \ldots}
Thank you.
\\
{\it (if two candidates, instead of ``those opposed'' say instead:} \\
All in favor of {\em (candidate two)\/} raise their right hand.
{\em \ldots}
Thank you.{\it )}

\end{chairsays}

\begin{chairsays}
{\em Candidate\/} is elected.
\end{chairsays}

\begin{chairsays}
I \emph{(or \emph{``The Chair''})} will entertain a motion to adjourn.

{\em Member:\/} So moved.
\end{chairsays}

\begin{chairsays}
Is there a second?

{\em Member:\/} Seconded.
\end{chairsays}

\begin{chairsays}
If there are no objections, this meeting will be adjourned.
{\em (Pause briefly.)}

Hearing no objections, \ldots {\em (pause again)} \ldots this meeting
is adjourned.
\end{chairsays}


\end{document}
