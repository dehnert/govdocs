\documentclass{bylaws}
\title{Tech Squares Constitution}
\date{2015}
\begin{document}
\maketitle

\begin{history}
Adopted 22 May 1984
Amended 19 January 1988
Amended 1 March 1988
Amended 16 August 1994
Amended 13 August 2002
Amended 23 April 2013
Amended 3 September 2013
Rewritten Spring 2015
\end{history}


\article{Name and Purpose}
\section This organization shall be called Tech Squares, the Square and Round Dance Club of the Massachusetts Institute of Technology, hereinafter referred to as \textit{the club} and \textit{MIT}, respectively.
\section The club shall regularly hold Modern Western Square Dances.
\section The club shall hold classes to introduce beginners to square dancing.


\article{Membership}
\section Upon graduation from the class, a person becomes a \textit{club member}.
\section A person who has not graduated from the class must dance club level to become a club member. Other requirements may be stated in the standing policies.
\section Membership shall include the right to wear the club badge. No one but club members may wear the club badge.
\section Any club member who meets the attendance requirements specified in the standing policies shall be a \textit{voting member}.
\section Only voting members may vote at club meetings or join the Executive Committee.
\section The club shall not discriminate based on any characteristic listed in the MIT Nondiscrimination Policy for membership, officer position, or in any other aspect.


\article{Dances}
\section The club level shall be the \textsc{Callerlab} Plus program.
\section The club shall hold \textit{weekly dances} year-round allowing exceptions for holidays, weather, and similar occasions. Club dances shall be at club level, except as required for the class.
\section The club may hold additional dances and workshops at club level or other levels.


\article{Class}
\section The club shall hold a class during each Fall and Spring academic term at MIT to introduce beginners to square dancing. The class shall teach club level.
\section The class shall be coordinated by the Class Coordinator.
\section The club caller shall be responsible for teaching the class.


\article{Employees}
\section \textit{Club employee} shall be defined as anyone who is paid for services rendered to the club.
\section No person may be an Executive Committee member and a club employee concurrently.
\section The club shall hire two permanent employees to provide calling and cueing at weekly dances: a square dance caller, called the \textit{club caller}, and a round dance cuer, called the \textit{club cuer}.
\section The club caller and the club cuer shall be members of the club, but not voting members, for the term of their employment.  They may wear the club badge.  They may attend club or Executive Committee meetings only at the invitation of the Executive Committee.
\section The selection of a club caller or club cuer must be approved by the club.
\section The Executive Committee may hire other callers and cuers.  Such employees do not become club members by fact of their employment by the club.


\article{Club Meetings}
\section The club shall regularly hold meetings, called \textit{club meetings}, to review the club status and conduct club business.
\section Club meetings shall be open to all club members except as otherwise noted in the constitution or standing policies.
\section A quorum shall consist of one-quarter of the voting membership and one-quarter of the voting members who are MIT students.
\section At club meetings, no business other than reports and discussions shall be carried on unless a quorum is present.
\section If fewer than 50\% of those voting are MIT students, then the votes of the MIT students shall be scaled so as to collectively constitute 50\% of the resulting total voting power.
\section A ``two-thirds vote'' refers to at least two-thirds of the voting power of those present and voting being cast in favor. Abstentions count as not voting. Similar language applies with other fractions.
\section Decisions of any club meeting shall be binding upon the Executive Committee
and the club officers.


\article{Executive Committee}
\section The club shall have an \textit{Executive Committee} (EC). The EC shall be responsible for the affairs of the club and is empowered to make such decisions as it feels are necessary to efficiently carry on the business of the club, except as otherwise governed by this constitution or the standing policies.
\section The members of the Tech Squares Executive Committee shall be the President, Vice President, Treasurer, Vice Treasurer (if applicable), Class Coordinator, Publicity Coordinator, and Members-at-Large.
\section A majority vote of the Executive Committee shall be necessary for any official decisions it makes.
\section The Executive Committee may appoint and remove other club officers, as it feels appropriate or as dictated by the standing policies. Appointment shall not grant membership on the Executive Committee.
\section The Executive Committee may overrule any officer, but shall exercise restraint in doing so.

\subarticle{Responsibilities}
\subsection The President and Vice President shall be responsible for the overall operation of the club.
\subsection The Treasurer and Vice Treasurer shall be responsible for the finances of the club.
\subsection The Class Coordinator shall be responsible for the organization of the class.
\subsection The Publicity Coordinator shall be responsible for organizing all class and club publicity.
\subsection Members-at-Large shall facilitate the business of the club, assist the other officers, and take on other duties as appropriate.

\subarticle{Vice Treasurer}
\subsection The position of Vice Treasurer will exist if requested. The Treasurer may request one immediately upon election or at any later point. The Executive Committee may also request a Vice Treasurer at any time by majority vote.
\subsection If a Vice Treasurer is requested during the election, the position will be filled as usual in that election. If a Vice Treasurer is requested at any other time, the position is filled as with any other vacancy on the Executive Committee.

\subarticle{Composition Restrictions}
\subsection In general, a person may hold multiple Executive Committee positions. However, the President, Vice President, Treasurer, and Vice Treasurer (if requested) must be distinct people. Those people may also hold additional Executive Committee positions.
\subsection Each distinct person shall have one vote as a member of the Executive Committee, even if they hold multiple officer positions.
\subsection The Executive Committee shall contain eight distinct people (unless there is no Vice Treasurer, in which case it shall contain seven). Enough Members-at-Large shall be elected to fill the Executive Committee.
\subsection The President and Treasurer must be MIT students, and a total of at least four Executive Committee members must be MIT students.


\article{Elections and Vacancies}

\subarticle{Normal Elections}
\subsection Elections of officers shall be held each year in the spring. Officers shall take office on the earlier of one week after the spring class ends or June 1. Elections shall be held four to eight weeks prior to the start of their term of office.
\subsection An officer may serve until their successor's term begins in the next spring (about one year), or they may choose to vacate their office at the end of fall term. Fall elections shall be held for positions that are being vacated. Officers elected in the fall take office on the earlier of one week after the fall class ends or January 1.
\subsection Elected officers must be voting members at the time of elections and have been club members for at least three months. The latter requirement may be waived by a two-thirds vote of the club. Advance notice is not required.

\subarticle{Vacancies}
\subsection If there is a vacancy in the office of President, the Vice President temporarily assumes the President's role.
\subsection In the case of a vacancy in another Executive Committee position, the President appoints a temporary officer.
\subsection In either case, an election is to be held within four weeks to appoint a replacement who will serve the remainder of their term of office.
% To satisfy constraints on EC composition, positions vacated by students may need to be filled by either another student or a member of the EC. 

\subarticle{Removals}
\subsection Upon petition by at least 20\% of the voting members of the club or one-half the members of the Executive Committee, the Executive Committee shall call a club meeting to discuss the removal of an officer.
\subsection The officer may be removed by a three-fourths vote.


\article{Amendments}
\section This constitution may be amended at a club meeting by a two-thirds vote.
\section A proposed amendment to the constitution must be submitted in writing to the Executive Committee. It shall publicize the proposed amendment to the club at least four weeks before it will be voted on.
\section Changes to the amendment can be made during those four weeks, including at the meeting, so long as the changes preserve the essential nature of the amendment.


\article{MIT Recognition}
\section The club shall be a recognized MIT student organization.
\section This constitution is the governing law of the club, except where it
conflicts with the rules of MIT.
\section The club agrees to abide by the rules and regulations of the
Association of Student Activities, and its executive board. This
constitution, amendments to it, and the standing policies of this organization
shall be subject to review by the ASA Executive Board to ensure that
they are in accordance with the aforementioned rules and regulations.

\end{document}
