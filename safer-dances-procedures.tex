\documentclass{article}
% html and hyperref both define \hyperref, but hyperref's seem to be better,
% so it goes second.
\usepackage{html} % LaTeX2HTML utilities
\usepackage{hyperref}

\newcommand{\baseurl}{\latexhtml{http://www.mit.edu/~tech-squares/govdocs/}{}}
\newcommand{\matchextension}{.\latexhtml{pdf}{html}}
\newcommand{\matchlink}[2]{\href{\baseurl#1\matchextension}{#2}}

\title{Safer Dances Procedures}
\author{Tech Squares}
\date{}
\hypersetup{pdftitle=Tech Squares Safer Dances Procedures}

% Margins
\oddsidemargin=.125in
\textwidth=6.25in
\topmargin=0in
\headheight=0in
\headsep=0in
\textheight=9in
\parskip=.4\baselineskip % blank space between paragraphs

\begin{document}

\maketitle

\section{Safer Dances Coordinators}

Each Executive Committee (EC) shall designate one or two Safer Dances Coordinator(s) to be the point person (people) for this policy. A member of the EC (such as the President or Vice President) is preferable, though the EC may instead appoint a member of the club with particular experience or interest in the subject.

The Safer Dances Coordinators shall generally:
\begin{itemize}
\item Be available to hear reports of violations of this policy (or concerning behavior) from club members and other attendees, whether in person or via email
\item Be available to speak with or informally warn those engaging in concerning behavior, on behalf of club members or other attendees
\item Collect reports made to other club officers, event organizers, etc.
\item Maintain records of past reports and sanctions
\item Serve as point person for enforcement actions
\end{itemize}


\section{Handling reports}

If you receive reports of harassment, please ask for (but do not pressure the reporters to provide) the following information, if it is not already included:

\begin{itemize}
\item Identifying information (name or description) of the person violating this policy
\item Specific behavior that was in violation or concerning
\item Approximate time of the behavior
\item Circumstances surrounding the incident
\item Other people involved in the incident
\end{itemize}

You may suggest contacting MIT officials or law enforcement (the Safer Dances Policy includes some suitable resources), but should not force them to.

Reports should be forwarded to the Safer Dances Coordinators as soon as possible - forwarding it that evening is preferred (so it doesn't get forgotten). (If you don't have time to type up the full report, it is better to send a preliminary report immediately, and type up the rest later. That way, if you forget the Safer Dances Coordinators can remind you.) If the reporter makes it clear whether they are hoping for action or merely reporting a potential pattern of behavior, you should provide that information as well. If the reporter specified any restrictions on how or with whom their report could be shared, you must respect that and pass those restrictions on to the Safer Dances Coordinators. 


\section{Immediate sanctions}

In the case of inappropriate behavior at an event, the organizers of the event or any EC member present can impose immediate sanctions. If there is no EC member or clear event organizer present, any appointed officer can impose sanctions as well. If multiple potential sanctioners are present, discussing it first is encouraged if time permits but not required. The EC member or other sanctioner may inform the subject themselves, or may delegate to another person present or the MIT Campus Police.

Sanctions may include:
\begin{itemize}
\item Formal warning, which should specify which behavior was problematic
\item Expulsion from the event
\item Banning from Tech Squares events for up to four days
\end{itemize}

The Safer Dances Coordinators should be informed of all sanctions, and the EC must be informed if the subject is banned from other events.


\section{Interim sanctions}

By majority vote (including via email), the EC may impose interim sanctions while they investigate an alleged violation of this policy. These are not intended to replace the long-term sanctions described below, but rather provide time for the EC to conduct a fair and prompt investigation without risking community members.

Interim sanctions may include:
\begin{itemize}
\item Formal warning/probation
\item Banning from some or all Tech Squares events until the conclusion of the investigation
    \begin{itemize}
    \item Events include regular Tuesday and Saturday dances, club meetings, and all other events using Tech Squares resources, such as rooms reserved through Tech Squares
    \end{itemize}
\item Relieving a club officer or other member of their club duties until the conclusion of the investigation
\end{itemize}

The EC should aim to complete the investigation within a few weeks, and no interim sanction may last longer than two months.


\section{Investigation}

The Safer Dances Coordinators and/or the EC should investigate all credible reports of violations of this policy. Such investigation should include talking with the complainant, respondent, and other victims or witnesses. Investigators may choose to meet with MIT offices for guidance or assistance, but are not expected to. The complainant and respondent must be given the opportunity to meet separately with the EC, but are not required to do so.


\section{Long-term sanctions}

By two-thirds majority vote of those voting, the EC may impose longer-term or more severe sanctions, after carrying out an investigation as described above. Such sanctions should be based principally on behavior at or in relation to Tech Squares events, although past history or history outside Tech Squares may contribute (such as in determining whether to ban somebody or merely warn them). Goals of sanctions include making community members feel safe at Tech Squares events, preventing future incidents, and encouraging offenders to improve their behavior.

Long-term sanctions may include:
\begin{itemize}
\item Formal warning/probation
\item Relieving a club officer or other member of their club duties
\item Banning from some or all Tech Squares events
    \begin{itemize}
    \item Events include regular Tuesday and Saturday dances, club meetings, and all other events using Tech Squares resources, such as rooms reserved through Tech Squares
    \item Permanent bans, with an opportunity to petition the EC to rescind them, may be appropriate for serious violations where the offender seems unlikely to improve their behavior or people are unwilling to be around them.
    \item Bans with a defined expiration time may be appropriate for relatively mild offenses where it seems likely the offender would improve, and a more serious penalty than a warning is desired to pressure them to do so.
    \end{itemize}
\item Revocation of membership
\end{itemize}


\section{Records}

The Safer Dances Coordinators shall maintain and pass down records of reported violations and sanctions. In general, records of immediate or interim sanctions should be maintained for three years. In general, records of reports should be kept for three years or not passed down, depending on the severity of the incident and wishes of the reporter and victim. Records of long term sanctions should

\begin{itemize}
\item Include basic details: respondent name, sanction imposed, basics of the offense
\item Have contact information for several people involved in the investigation (most likely EC members) who have more complete information (complainant, witnesses, evidence, details of reasoning, etc.)
\item Be tracked for the duration of the sanction and six years after they expire
\end{itemize}

The Safer Dances Coordinators shall provide to all EC members and event organizers who are likely to be impacted the current list of people banned from Tech Squares events. In the interests of confidentiality (both of those sanctioned and others involved in the case), the list of people banned should not be widely distributed.

Records of reports and sanctions should generally be closely held. If patterns of behavior are suspected or an investigation is started, the Safer Dances Coordinators may share relevant information with select EC members, event organizers, or other officers. Everyone should strive at all times to respect the privacy interests, explicit and implicit, involved in the records and information they are privy to.


\section{Publication of policy}

Each class shall be informed of this policy and the reporting mechanisms early in the class (such as in the week 1 handout). New club members (class graduates and those directly granted membership by the EC) shall be sent this policy upon joining the club.

The full club shall be reminded of this policy at least once a year, and the policy shall be published on the Tech Squares website.


\section{Notes}

The Geek Feminism wiki has many pages about anti-harassment policies, starting with\linebreak\url{http://geekfeminism.wikia.com/wiki/Conference_anti-harassment}. In particular, their \href{http://geekfeminism.wikia.com/wiki/Conference_anti-harassment/Policy}{sample policy} was a starting point for this policy, our Safer Dances Coordinator is similar to their \href{http://geekfeminism.wikia.com/wiki/Conference_anti-harassment/Duty_officer}{duty officer}, and the section on handling reports is derived from \href{http://geekfeminism.wikia.com/wiki/Conference_anti-harassment/Responding_to_reports}{their page}. Their \href{http://geekfeminism.wikia.com/wiki/Why_anti-harassment_policies_should_be_public}{``Why anti-harassment policies should be public''} has some comments on why conferences should have public policies (rather than secret policies), most of which are relevant to us; in addition to the reasons they list, our policy can also provide guidance to the officers, giving them one fewer thing to figure out on the fly if something gets reported. We also drew on the policies of the \href{https://docs.google.com/document/d/1hh9mXPxmazx8r1W1YKEzOWGwC3iTGX0qyiQOEYzFk9Q/pub}{Melbourne Lindy Exchange} and \href{http://swingdancesydney.com/codeofconduct.html}{Swing Dance Sydney}.

\end{document}
